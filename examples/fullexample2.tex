\documentclass{article}
\usepackage[utf8]{inputenc}
\usepackage{amsmath}

\begin{document}

    %--------------------------------------------------------------------------------
    \section{First example}

    The well known Pythagorean theorem \(x^2 + y^2 = z^2\) was
    proved to be invalid for other exponents.
    Meaning the next equation has no integer solutions:

    \(x^n + y^n = z^n \)


    %--------------------------------------------------------------------------------

    \section{Second example}

    In physics, the mass energy equivalence is stated by the equation $E=mc^2$, discovered by Albert Einstein.

    The mass    energy equivalence is described by the famous equation
    \[ E=mc^2 \]
    discovered by Albert Einstein.
    In natural units ($c$ = 1), the formula expresses the identity
    \begin{equation}
        E=m
    \end{equation}

    \section{Third example}

    This is a simple mathematic expression \(\sqrt{x^2+1}\) inside text.
    And this is also the same:
    \begin{math}
        \sqrt{x^2+1}
    \end{math}
    but by using another command.

    This is a simple mathematic expression without numbering
    \[\sqrt{x^2+1}\]
    separated from text.

    This is also the same:
    \begin{displaymath}
        \sqrt{x^2+1}
    \end{displaymath}

    ...and this:
    \begin{equation*}
        \sqrt{x^2+1}
    \end{equation*}

\end{document}
